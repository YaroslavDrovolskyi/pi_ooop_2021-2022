\label{index_md_glossary}%
\Hypertarget{index_md_glossary}%
 {\bfseries{\mbox{\hyperlink{class_figure}{Figure}} (фігура)}} – шахова фігура (пішка, король і тд). В деякій літературі зустрічається синонім {\itshape piece}.

{\bfseries{Fig\+Type}} – тип фігури. ~\newline
 Фігури бувають таких {\itshape типів}\+:
\begin{DoxyItemize}
\item {\itshape pawn} – пішка;
\item {\itshape horse} -\/ кінь;
\item {\itshape bishop} – слон (фігура, яка ходить по діагоналях);
\item {\itshape rook} – тура (ходить по вертикалі та по горизонталі);
\item {\itshape queen} – королева;
\item {\itshape king} – король.
\end{DoxyItemize}

{\bfseries{\mbox{\hyperlink{struct_point}{Point}}}} – координата клітинки на шаховій дошці. Початок відліку від лівого нижнього кута. Він починається з точки (0;0).

{\bfseries{\mbox{\hyperlink{struct_move}{Move}}}} – хід фігурою. Містить початкову координату (from) та кінцеву (dest).

{\bfseries{\mbox{\hyperlink{class_field}{Field}} (chess board)}} – шахова дошка.

{\bfseries{\mbox{\hyperlink{struct_cell}{Cell}}}} – клітинка на шаховій дошці.

{\bfseries{\mbox{\hyperlink{class_army}{Army}}}} – набір шахових фігур одного кольору.

{\bfseries{Marks}} – це позначки (виділення тих клітинок), куди може зробити хід вибрана фігура. Також є позначка, яка позначає клітинку, в якій знаходиться вибрана фігура.

{\bfseries{Moves history / moves list}} -\/ структура, яка зберігає інформацію про зроблені ходи та \char`\"{}знищені\char`\"{} фігури впродовж гри. 