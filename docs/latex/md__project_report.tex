Звіт-\/ретроперспектива до лабораторної роботи №3 та проекту Дровольського Ярослава, К-\/29 У проекті я реалізував гру «Шахи» людини з комп’ютером через графічний інтерфейс. У процесі написання гри я використовував бібліотеку SFML.
\begin{DoxyEnumerate}
\item Які конкретні задачі планували вирішувати за допомогою цієї бібліотеки?
\end{DoxyEnumerate}

SFML я використовував для написання графічного інтерфейсу та графічного представлення фігур і дошки.


\begin{DoxyEnumerate}
\item Чому було обрано саме цю бібліотеку, а не аналоги?
\end{DoxyEnumerate}

Я використав SFML через те, що вона надає всі необхідні для мого проекту інструменти (спрайти, обробка подій тощо). Крім цього, за допомогою SFML можна відносно швидко написати графіку – а це дає можливість сконцентруватися на логіці гри.


\begin{DoxyEnumerate}
\item Наскільки просто та зрозуміло було отримати, встановити, налаштувати та почати використовувати цю бібліотеку?
\end{DoxyEnumerate}

Я працював у Visual Studio і використав Nu\+Get Manager, тому для встановлення та налаштування бібліотеки потрібно було завантажити SFML (через Nu\+Get) та змінити деякі налаштування проекту. (\href{https://www.youtube.com/watch?v=on7U-90gfrI}{\texttt{ https\+://www.\+youtube.\+com/watch?v=on7\+U-\/90gfrI}})


\begin{DoxyEnumerate}
\item Наскільки зрозумілою та корисною була документація бібліотеки?
\end{DoxyEnumerate}

Документація SFML досить зрозуміла\+: для кожного класу є список всіх методів та їх короткий опис. Я звертався до документації, коли потрібно було дізнатися, що робить той чи інший метод або детальніше подивитися, які параметри він приймає. Загалом, документація для мене була корисною. Також, окрім документації, я звертався до перекладу офіційного tutorial (\href{https://habr.com/ru/post/278977/}{\texttt{ https\+://habr.\+com/ru/post/278977/}}).


\begin{DoxyEnumerate}
\item Наскільки було зрозуміло, як саме використовувати бібліотеку, які класи/методи/функції використовувати для вирішення поставлених задач?
\end{DoxyEnumerate}

В більшості випадків призначення класів/методів були інтуїтивно зрозумілі. Якщо у мене виникали питання щодо класів/методів, я читав документацію та шукав інформацію в Інтернеті – це і давало відповідь на моє запитання.


\begin{DoxyEnumerate}
\item Наскільки зручно було використовувати бібліотеку, чи не треба було писати багато надлишкового коду?
\end{DoxyEnumerate}

Загалом використовувати бібліотеку було зручно, але потрібно було писати деякі стандартні рядки для певних задач. Наприклад, щоб створити спрайт, потрібно було спочатку створити об’єкт класу Texture, потім завантажити в нього текстуру, потім створити об’єкт класу Sprite, а далі за необхідності вказати частину текстури для відображення та масштаб. Такі послідовності рядків легко знайти в tutorial чи в Інтернеті. А далі під час активного використання такий код поступово стає зрозумілим і проблем не виникає.


\begin{DoxyEnumerate}
\item Наскільки зрозумілою була поведінка класів/методів/функцій з бібліотеки?
\end{DoxyEnumerate}

Класи та методи в SFML мають інтуїтивні назви, тому особливих проблем з розумінням поведінки не виникало. Хоча були нюанси, про які треба було читати (наприклад поведінка об’єктів класу Event).


\begin{DoxyEnumerate}
\item Наскільки зрозумілою була взаємодія між різними класами/методами/функціями цієї бібліотеки, а також взаємодія між бібліотекою та власним кодом?
\end{DoxyEnumerate}

Взаємодія між SFML та моїм кодом прозора\+: в обробниках подій розглядаємо можливі варіанти (наприклад, в обробнику подій для кліку мишкою ми розглядали варіанти, де відбувся клік) і викликаємо відповідну функцію, яка робить потрібні зміни в дошці (в структурі даних). А вже потім, незалежно від того, які зміни відбулися і чи вони взагалі відбулися, на кожній ітерації game loop викликається метод update(), який відмальовує вікно.

Взаємодія класів SFML один з одним складалася з таких послідовностей коду, які описані у попередньому пункті.

Я старався винести головний ігровий цикл, обробники подій, прораховування ходів, механізм «биття» фігури, ініціалізацію дошки з функції main() у окремий клас Game. Для того, щоб запустити гру потрібно лише створити об’єкт класу Game та викликати в нього метод exec().


\begin{DoxyEnumerate}
\item Чи виникали якісь проблеми з використанням бібліотеки? Чи вдалось їх вирішити, як саме?
\end{DoxyEnumerate}

Під час використання бібліотеки в мене виникла проблема з тим, що при наведенні курсора на деякі об’єкти відбувалися графічні артефакти. Після додавання синхронізації частот екрану і програми (set\+Vertical\+Sync\+Enabled(true)) проблема була вирішена. В цілому, з використанням бібліотеки не виникало особливих проблем, бо вона досить зрозуміла + є якісна документація та офіційний tutorial.


\begin{DoxyEnumerate}
\item Що хорошого можна сказати про цю бібліотеку, які були позитивні аспекти використання бібліотеки?
\end{DoxyEnumerate}

SFML має такі позитивні аспекти\+:
\begin{DoxyItemize}
\item Прозорість класів, їхньої поведінки, методів
\item Зрозуміла документація
\item Достатня кількість посібників, статей, відеоуроків в Інтернеті
\item Якщо є якісь запитання, як саме щось реалізувати, то відповіді на них можна знайти на Stack\+Overflow чи на подібних сайтах
\item Потрібно відносно мало коду, щоб створити першу програму, яка має назву «\+SFML Works!» (аналог «\+Hello world»).Причому цей код є на офіційному сайті (\href{https://www.sfml-dev.org/documentation/2.5.1/index.php}{\texttt{ https\+://www.\+sfml-\/dev.\+org/documentation/2.\+5.\+1/index.\+php}})
\end{DoxyItemize}
\begin{DoxyEnumerate}
\item Що поганого можна сказати про цю бібліотеку, які були негативні аспекти використання бібліотеки?
\end{DoxyEnumerate}

Єдиний такий аспект – це те, що створення графічних об’єктів та їхнє відмалювання треба робити вручну\+: спочатку його створити, задати позицію, потім відобразити за допомогою window.\+draw().


\begin{DoxyEnumerate}
\item Якби довелось вирішувати аналогічну задачу, але вже враховуючи досвід використання в цій лабораторній роботі, що варто було б робити так само, а що змінити? Можливо, використати інші бібліотеки, чи використати інші можливості цієї бібліотеки, чи інакше організувати код, чи ще щось?
\end{DoxyEnumerate}

На мою думку, можна було б розбити код на більшу кількість файлів для покращення читабельності. 